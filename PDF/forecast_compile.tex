\documentclass[11pt]{article}
%%%%%%%%%%%%%%%%%%%%%%%%%%%%%%%%%%%%%%%%%%%%%%%%%%%%%%%%%%%%%%%%%%%%%%%%%%%%%%%%%%%%%%%%%%%%%%%%%%%%%%%%%%%%%%%%%%%%%%%%%%%%%%%%%%%%%%%%%%%%%%%%%%%%%%%%%%%%%%%%%%%%%%%%%%%%%%%%%%%%%%%%%%%%%%%%%%%%%%%%%%%%%%%%%%%%%%%%%%%%%%%%%%%%%%%%%%%%%%%%%%%%%%%%%%%%
\usepackage{amssymb}
\usepackage{amsfonts}
\usepackage{graphicx}
\usepackage{amsmath}
\usepackage{rotating}
\usepackage{epsfig}
\usepackage{booktabs}
\usepackage{natbib}
\usepackage[onehalfspacing]{setspace}
\usepackage{appendix}
\usepackage{rotating}
\usepackage{multirow}
\usepackage{bbm}
\usepackage{color}
\usepackage{lscape}
\usepackage[FIGTOPCAP]{subfigure}
\usepackage{anysize}
\usepackage[pdfborder={0 0 0}]{hyperref}
\usepackage{placeins}
\usepackage{versions}
%\usepackage{longtable}
\usepackage{varwidth}
\usepackage{eurosym}

\setcounter{MaxMatrixCols}{10}


\pagestyle{myheadings}
\parindent10pt
\parskip1ex plus1.5ex minus0.2ex
\setcounter{secnumdepth}{5}
\setcounter{tocdepth}{2}
\voffset0cm
\topmargin-0.00in
\oddsidemargin-1.5cm
\evensidemargin0cm
\textheight10.5in
\textwidth7.5in
\renewcommand{\baselinestretch}{1.3}
\newcommand{\bc}{\begin{center}}
	\newcommand{\ec}{\end{center}}
\newcommand{\be}{\begin{equation}}
\newcommand{\ee}{\end{equation}}
\newcommand{\bea}{\begin{eqnarray}}
\newcommand{\eea}{\end{eqnarray}}
\newcommand{\bean}{\begin{eqnarray*}}
	\newcommand{\eean}{\end{eqnarray*}}
\newcommand{\refline}{\raisebox{1ex}{\underline{\hspace{2cm}} \quad}}
\newtheorem{theorem}{Theorem}
\newtheorem{definition}{Definition}
\newtheorem{lemma}{Lemma}
\newtheorem{corollary}{Corollary}
\newtheorem{proposition}{Proposition}
\newtheorem{assumption}{Assumption}
\newcounter{pkt}
\newenvironment{tlist}{\begin{list}{(\roman{pkt})}{\usecounter{pkt}\parskip0ex\parsep0ex\itemsep0ex\topsep0ex}}{\end{list}}
\def\EE{\mathord{I\kern-.35em E}}
\def\PP{\mathord{I\kern-.3em P}}
\def\QQ{\mathord{Q\kern-5pt\hbox{\raise1.1pt\hbox{\vrule height5pt}}\kern5pt}}
\def\RR{\mathord{I\kern-.3em R}}

\newcommand{\highlight}[1]{\colorbox{yellow}{#1}}

\def\changemargin#1#2{\list{}{\rightmargin#2\leftmargin#1}\item[]}
\let\endchangemargin=\endlist 

\begin{document}


\begin{figure}[t!]	
	\caption{Forecasts comparison}
	\begin{center}	
			\begin{tabular}{ccc}	
				 {\footnotesize No 2020:M3-M6 (E2)}  &
				{\footnotesize Full Sample (E3)}  & {\footnotesize Lenza \& Primiceri}\\ [1ex]	
				 \multicolumn{3}{c}{GDP} \\[1ex]
				\includegraphics[clip, trim=1cm 0.0cm 0.0cm 0.0cm, width=2.15in,height=1.5in]{../Figures/cgr2_7_9.eps} &
				\includegraphics[clip, trim=1cm 0.0cm 0.0cm 0.0cm, width=2.15in,height=1.5in]{../Figures/cgr1_7_9.eps} &
				\includegraphics[clip, trim=1cm 0.0cm 0.0cm 0.0cm, width=2.15in,height=1.5in]{../Figures/cgr3_7_9.eps} \\[1ex]
				 \multicolumn{3}{c}{Unemployment} \\[1ex]		
				\includegraphics[clip, trim=1cm 0.0cm 0.0cm 0.0cm, width=2.15in,height=1.5in]{../Figures/lv2_7_1.eps} &
				\includegraphics[clip, trim=1cm 0.0cm 0.0cm 0.0cm, width=2.15in,height=1.5in]{../Figures/lv1_7_1.eps} &
				\includegraphics[clip, trim=1cm 0.0cm 0.0cm 0.0cm, width=2.15in,height=1.5in]{../Figures/lv3_7_1.eps} \\[1ex]
				 \multicolumn{3}{c}{Inflation} \\[1ex]	
				\includegraphics[clip, trim=1cm 0.0cm 0.0cm 0.0cm, width=2.15in,height=1.5in]{../Figures/gr2_7_3.eps} &
				\includegraphics[clip, trim=1cm 0.0cm 0.0cm 0.0cm, width=2.15in,height=1.5in]{../Figures/gr1_7_3.eps} &
				\includegraphics[clip, trim=1cm 0.0cm 0.0cm 0.0cm, width=2.15in,height=1.5in]{../Figures/gr3_7_3.eps} 
			\end{tabular}
	\end{center}
	{\footnotesize\emph{Notes:} We forecast quarterly averages. Actual values (solid red, January 2022 vintage) and forecasts: median (solid black) and 90\% bands (light grey) constructed from the posterior predictive distribution. Green squares represent median forecasts from the SPF. For GDP we depict percentage change relative to December 2019.  No 2020:M3-M6 (E2): we treat monthly observations from 2020:M3 to M6 and quarterly observations for 2020:Q1 and Q2 as missing. Full Sample (E3): the MF-VAR is estimated based on data available at the forecast origin. Filtering uses the vintage available at the forecast origin (F1).
	}\setlength{\baselineskip}{4mm}
\end{figure}





\end{document}



